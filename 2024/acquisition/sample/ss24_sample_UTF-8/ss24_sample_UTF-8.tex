%%%%%%%%%%%%%%%%%%%%%%%%%%%%%%%%%%%%%%%%%%%%%%%%%%%%%%%%%%%%%%%%%%%%%%%%%%%%%
%集録提出用のサンプルファイルです。
%このファイルを参考にして集録を作成してください。
%%%%%%%%%%%%%%%%%%%%%%%%%%%%%%%%%%%%%%%%%%%%%%%%%%%%%%%%%%%%%%%%%%%%%%%%%%%%%
%%%%%%%%%%%%%%%%%%%%%%%%%%%%%%%%%%%%%%%%%%%%%%%%%%%%%%%%%%%%%%%%%%%%%%%%%%%%%
\documentclass[a4paper,10pt,oneside,twocolumn,notitlepage,final]{jarticle}
\usepackage{ss24_UTF-8}
%%%%%%%%%%%%%%%%%%%%%%%%%%%%%%%%%%%%%%%%%%%%%%%%%%%%%%%%%%%%%%%%%%%%%%%%%%%%%
%%styファイルはtexファイルと同じディレクトリに置いてください。
%%ヘッダ、フッターなどの全体のレイアウトは変更しないでください。
%%ここより上は変更しないでください。
%%%%%%%%%%%%%%%%%%%%%%%%%%%%%%%%%%%%%%%%%%%%%%%%%%%%%%%%%%%%%%%%%%%%%%%%%%%%%
%%使いたいパッケージがある場合は以下に書いてください。

\usepackage[dvipdfmx]{graphicx}% 

%%BibTeXで参考文献を書く場合
\usepackage{natbib}
\bibpunct{(}{)}{;}{a}{}{,}


%%%%%%%%%%%%%%%%%%%%%%%%%%%%%%%%%%%%%%%%%%%%%%%%%%%%%%%%%%%%%%%%%%%%%%%%%%%%%
%%名前、所属、タイトルは以下に記入してください。
%%所属は以下のように()内にお願いします。
\author{苗字 名前 (東北大学大学院 ××研究科)}
\title{タイトルを記入してください}
%%%%%%%%%%%%%%%%%%%%%%%%%%%%%%%%%%%%%%%%%%%%%%%%%%%%%%%%%%%%%%%%%%%%%%%%%%%%%


\begin{document}
%%PDFの全ページ数が4ページ以内、容量が5MB以下に収まるようにしてください。
%%下限ページ数は設定しておりません。


%%概要は\abst内に記入してください。
%%\maketitleは必要ありません。
%%以下の\abst{}に概要を記入することにより
%%タイトル、名前、日付、概要が一括して出力されます。
\abst{
概要を記入してください。概要を記入してください。概要を記入してください。
概要を記入してください。概要を記入してください。概要を記入してください。
\\
概要を記入してください。概要を記入してください。概要を記入してください。
概要を記入してください。概要を記入してください。概要を記入してください。
概要を記入してください。概要を記入してください。概要を記入してください。
}



\section{Introduction}
セクションの名前は章立ての目安ですので適宜変更してください。



\section{Methods/Instruments \\ and Observations}
本文を記入してください。本文を記入してください。本文を記入してください。本文を記入してください。本文を記入してください。本文を記入してください。


\section{Results}
本文を記入してください。本文を記入してください。本文を記入してください。本文を記入してください。本文を記入してください。本文を記入してください。


\section{Discussion}
本文を記入してください。本文を記入してください。本文を記入してください。本文を記入してください。本文を記入してください。本文を記入してください。


\section{Conclusion}
本文を記入してください。本文を記入してください。本文を記入してください。本文を記入してください。本文を記入してください。本文を記入してください。


\section{図の挿入方法}
図の挿入方法の簡単な説明です。
figure環境で図を挿入します。
[htbp]で図を入れる位置を指定します。
例のように同時に書いている場合は
h:その場所、t:ページ上部、b:ページ下部、p:独立したページ
の順に挿入しようとします。


\begin{figure}[htbp]
 \centering
 \includegraphics[width=3cm,clip]{pic.pdf}
 \caption{図の挿入方法の説明}
 \label{pic} %ラベルをつけることにより参照できるようになります。
\end{figure}

\verb+\+includegraphics[width=5cm,clip]\{pic.pdf\}
で同じディレクトリにある"pic.pdf"という画像ファイルを指定します。
オプションで画像の大きさを指定します。(epsファイルも同様)

\verb+\+figref\{ラベル\}でこのように参照できます(例: \figref{pic})。

\section{表の挿入方法}
表の挿入方法の簡単な説明です。
table環境とtabular環境で表を挿入します。
table環境のオプションで表の位置を決めます。
この方法は図の位置と同じです。
table環境の後に列を決めます。
\{lcr\}のようにすれば一列目は左寄せ、二列目はセンタリング、三列目は右寄せになります。
サンプルでは\{l|c|r\}のようにして列に罫線を入れています。
\verb+\+hline を入れることにより行の罫線を入れることができます。

\begin{table}[htb]
 \caption{表の挿入方法の説明}
  \centering
   \begin{tabular}{|l|c|r|} \hline
     A & B & C  \\ \hline
     1A & 1B & 1C  \\
     2A & 2B & 2C  \\ \hline
   \end{tabular}
  \label{table}%ラベルをつけることにより参照できるようになります。
\end{table}

\verb+\+tabref\{ラベル\}でこのように参照できます(例: \tabref{table})。

\section{参考文献の書き方}
\noindent
著者が一人の場合\citep{ラベル1}。\\
著者が二人の場合\citep{ラベル2}。\\
著者が三人の場合\citep{ラベル3}。\\
\section*{Acknowledgement}
謝辞がある場合は記入してください。

\small
\begin{thebibliography}{99}
\bibitem[著者A(2013)]{ラベル1}
 著者A 2013, 発行元1
 \bibitem[著者B \& 著者C(2014)]{ラベル2}
 著者B, \& 著者C 2014,  発行元2
\bibitem[著者D et al.(2015)]{ラベル3}
 著者D, 著者E, \& 著者F 2015, 発行元3
 
 
\end{thebibliography}






\end{document}
